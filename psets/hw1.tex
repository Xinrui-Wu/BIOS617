\documentclass[12pt]{article}
\usepackage{graphicx}
\usepackage[margin=1.0in]{geometry}

\def\given{\, | \,}

\begin{document}

\title{Homework \# 1}
\author{BIOS 617}

\maketitle

\begin{enumerate}
\item Exercise in relationship between marginal and conditional means and variances.

Consider worker income $X$ from four departments: $C_1, C_2, C_3$, and $C_4$ with $3, 2, 4$, and $3$ employees respectively:

\begin{table}[!th]
\centering
\begin{tabular}{ c c c c}
$C_1$ & $C_2$ & $C_3$ & $C_4$ \\ \hline
50 & 90 & 18 & 32 \\
60 & 105 & 23 & 10 \\
70 & - & 22 & 60 \\
- & - & 25 & - \\ \hline
\end{tabular}
\end{table}

	\begin{enumerate}
		\item Find the mean $E(X)$ and variance $V(X)$ of a {\bf single} individual's income sampled at random. {\bf [10 pt]}
		\item Fill in the following table:
		\begin{table}[!th]
		\centering
		\begin{tabular}{c | c c c c}
		& $C_1$ & $C_2$ & $C_3$ & $C_4$ \\ \hline
		$E(X \given C)$ & - & - & - & - \\
		$V(X \given C)$ & - & - & - & - \\
		$P(C)$ & - & - & - & - \\
		\end{tabular}
		\end{table}

		where $E ( X \given C )$ is the mean income in each department, $V ( X \given C )$ is the variance of the income in each department, and $P ( C )$ is the probability that an individual selected at random from the population will belong to that department. {\bf [9 pt]}

		\item Show that $E (X) = E(E(X \given C) )$ and $V(X) = V(E(X \given C)) + E(V(X \given C))$ {\bf [10 pt]}

		\end{enumerate}

\item Suppose a population has mean~$\bar Y = 3$.  Let's assume the
Consider two subpopulations with means~$\bar Y_1 = 1.5$ and~$\bar Y_2 = 4$ respectively.  Suppose the joint population has mean~$x$

\item Consider a small population of $N = 8$ students with

\item Returning to the population in \# 3, consider a stratified sampling in which the population is stratified into low score ($<= 75$) and high score ($> 75$) strata.
	\begin{enumerate}
		\item Identify all possible random samples of size 4 obtained by selecting $n_h = 2$ from each stratum and make a historgram of the resulting means. {\bf [10 pt] }
		\item Compute the sampling variance of the sample means in (a) directly, and compare this to the sampling variance given by $\sum_{h=1}^2 P_h^2 \frac{1-f_h}{n_h} S_h^2$. {\bf [5 pt] }
	\end{enumerate}

\end{enumerate}
\end{document}