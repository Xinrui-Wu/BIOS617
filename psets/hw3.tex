\documentclass[12pt]{article}
\usepackage{graphicx}
\usepackage[margin=1.0in]{geometry}
\usepackage{hyperref}

\def\given{\, | \,}

\begin{document}

\title{Homework \# 3, Due February 24, 2020}
\author{BIOS 617}
\date{Assigned on: February 10, 2020}

\maketitle

\begin{enumerate}
\setlength{\itemsep}{15pt}%
\setlength{\parskip}{15pt}%

\item  The dataset `hw3.dat' contains a population of 50 individuals in a firm. The first column is an indicator of gender (1 for female, 0 for male) and the second for salaried (=1) versus hourly (=0).
	\begin{enumerate}
	\item For a systematic sample of size $n=10$, compute the (true) variance of the estimated proportion of women and of salaried workers. {\bf [10 pts]}
	\item Compute the variances under SRS for proportion of women and of salaried workers and the resulting design effects. {\bf [10 pts]}
	\item Comment on your findings in (b). {\bf [5 pt]}
	\end{enumerate}

\item Cochran 10.10 {\bf [20 pt]} Show that if $S_u^2 >0$, in the notation of section 10.6, a simple random sample of $n$ primary units, with $1$ element chosen per unit, is more precise than a simple random sample of $n$ elements ($n>1$, $M>1$). Show that the precision of the two methods is equal if $n/N$ is negligible. Would you expect this intuitively? \emph{Although this results can be shown exactly, you can make the assumption that $K$ is large, so that $\frac{K-1}{K} \approx 1$ and thus $\frac{N-1}{N} \approx 1$ to simplify calculations.}

\item A researcher took an SRS of 4 high schools from a region of 29 high schools for a study on the prevalence of routine vaping among high school students in the region:

\begin{table}[!th]
\centering
\begin{tabular}{p{1in} | p{1in} p{1in} p{1in}}
School & Number of students & Number of students interviewed & Number of routine vapers \\ \hline
1 & 800 & 20 & 10 \\
2 & 760 & 19 & 4 \\
3 & 800 & 20 & 6 \\
4 & 840 & 21 & 13 \\ \hline
\end{tabular}
\end{table}

	\begin{enumerate}
		\item Estimate the percentage of students who routinely vape in this region, along with a 95\% confidence interval. You may assume equal clusters and sample sizes corresponding to the mean cluster size and sample size in the four schools. {\bf [10 pt]}
		\item The researcher now wants to study the prevalence of vaping among high school students in a different region with 35 high schools. She will sample $k$ schools and interview $m$ students at each school, and can afford to spend 300 hours on the project.  Assuming that it will take 20 hours to recruit a school into the sample and travel to the school and 30 minutes to interview each student (accounting for no shows and other administrative tasks), propose values of $k$ and $m$ for an optimal sample design using the results from (a). {\bf [15 pt]}
	\end{enumerate}

\item We have the following population of size $N=5$: $(y_1 = 2, x_1 = 20)$, $(y_2 = 3, x_2 = 25)$, $(y_3 = 0, x_3 = 5)$, $(y_4 = 1, x_4 = 15)$, $(y_5 = 2, x_5 = 15)$
	\begin{enumerate}
		\item Determine all the possible SRS (without replacement) of size $n=2$ and compute the population variance of $r$. {\bf [10 pt]}
		\item Compute $V(r)$ approximated by $\frac{1-f}{n \bar X^2} \left[ S_y^2 + R^2 S_x^2 - 2 R S_{xy} \right]$.  Why is there a discrepancy with your results in (a)? Note: Stating that ``(b) is an approximation of (a)'' is an insufficient answer. {\bf [10 pt]}
	\end{enumerate}
\item The dataset \emph{mymu284.txt} contains 20 randomly-sampled observations from a set of 284 Swedish municipalities (see \url{https://rdrr.io/cran/sampling/man/MU284.html} for more details on this dataset). Given that the total value of the 1984 assessments was 874,017,000,000 kroner across all municipalities, compute an estimate and associated 95\% confidence interval for the total 1985 municipal taxation.

\end{enumerate}
\end{document}